\documentclass{article}
        \title{Pokemon Attack Simulator}
        \author{Anna Edgar}
        \date{June 10, 2013}
        \usepackage{listings}
        \usepackage{color}

        \definecolor{dkgreen}{rgb}{0,0.6,0}
        \definecolor{gray}{rgb}{0.5,0.5,0.5}
        \definecolor{mauve}{rgb}{0.58,0,0.82}

        \lstset{frame=tb,
          language=Python,
          aboveskip=3mm,
          belowskip=3mm,
          showstringspaces=false,
          columns=flexible,
          basicstyle={\small\ttfamily},
          numbers=none,
          numberstyle=\tiny\color{gray},
          keywordstyle=\color{blue},
          commentstyle=\color{dkgreen},
          stringstyle=\color{mauve},
          breaklines=true,
          breakatwhitespace=true
          tabsize=3
}
        \begin{document}
        \maketitle
        \section{Introduction}
        Pokemon is a turn based battle game where the user raises and battles creatures name Pokemon. Each Pokemon has various attributes that determine how well they will fare in battle and may learn up to 4 attacks. These attacks also have attributes that will help determine how much damage the Pokemon will deal. Even though the total amount of damage each attack will do is easy to determine, it does require a lot of information. The attacking Pokemon's attack and level attributes, the defending Pokemon's defense attribute, and the attack's power attribute all help determine the total amount of damage done. I have created this attack simulator to allow the user to determine the total damage an attack will use. This will allow the user to make more educated moves and better strategize.
        \subsection{Using XML}
        When creating a Pokemon quite a few details are needed. In order to speed up the process for the user the information on Pokemon and attacks will be saved to an XML document. This also allows multiple users to build off of each other's work. Ideally, many people would use this simulator and eventually a comprehensive list of all Pokemon would be built. Then a new user would never have to take time to input all of the details for a Pokemon. Reading and writing to the XML does take some additional computation time. However, the benefit of decreased user input outweighs the additional computation time. Ideally, a database would be used to store this information, but for the purpose of this assignment an XML document will suffice.
        \section{The Attack Class}
        An Attack object represents an attack that a Pokemon may use in battle. Each object must have a name, element, power, and accuracy.
        \subsection{Attack Methods}
        \begin{itemize}
            \item \textbf{init(self, name)}\\
            \tab{This method will create a new instance of an item that has already been added to the XML. The name parameter must be a string.}
            \item \textbf{create(name, element, power, accuracy, special = False )}\\
            \tab{This method add a new attack to the XML document. Some attacks are considered special attacks and will use the Pokemon's special attack and special defense attributes instead of the standard attack and defense attribute.}
        \end{itemize}
        \end{document}